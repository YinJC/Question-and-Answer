\usepackage[UTF8]{ctex}
%\ctexset{
%chapter/name = {第,章},
%%number = \chinese{chapter},
%}

%-=-=-=-=-=-=-=-=-=-=-=-=-=-=-=-=-=-=-=-=-=-=-=-=
%	PACKAGES
%-=-=-=-=-=-=-=-=-=-=-=-=-=-=-=-=-=-=-=-=-=-=-=-=

\usepackage[top=3cm,bottom=3cm,left=3cm,right=3cm,headsep=10pt,a4paper]{geometry} % Page margins

\usepackage{graphicx} % Required for including pictures
\graphicspath{{Pictures/}}
\usepackage{lipsum} % Inserts dummy text
\usepackage{subfig}
\usepackage{tikz} % Required for drawing custom shapes
\usepackage{tkz-euclide}
%\usepackage[english]{babel} % English language/hyphenation

\usepackage{enumitem} % Customize lists
\setlist{nolistsep} % Reduce spacing between bullet points and numbered lists

\usepackage{booktabs} % Required for nicer horizontal rules in tables

%\usepackage{sagetex}

\usepackage{subfiles} % Required for use with SageMath Cloud

\usepackage{xcolor} % Required for specifying colors by name
\definecolor{ocre}{RGB}{0,126,196} % Define the orange color used for highlighting throughout the book

%-=-=-=-=-=-=-=-=-=-=-=-=-=-=-=-=-=-=-=-=-=-=-=-=
%	MATHEMATICS SETTINGS
%-=-=-=-=-=-=-=-=-=-=-=-=-=-=-=-=-=-=-=-=-=-=-=-=

\usepackage{
amsmath,
amssymb,
amsfonts,
amsthm,
cancel,
colortbl,
longtable,
pdflscape,
setspace,
mathtools,
pgfplots
}
\usepackage{makecell}
%-=-=-=-=-=-=-=-=-=-=-=-=-=-=-=-=-=-=-=-=-=-=-=-=
%	STOCKHOLMS COLOR THEME 20151012-071009
%-=-=-=-=-=-=-=-=-=-=-=-=-=-=-=-=-=-=-=-=-=-=-=-=

%== BLUE
	\definecolor{sthlmLightBlue}{RGB}{214,237,252} % HEX #d6edfc
		\newcommand{\csthlmLightBlue}[1]{{\color{sthlmLightBlue}{#1}}}
	\definecolor{sthlmBlue}{RGB}{0,110,191} % HEX #006ebf
		\newcommand{\csthlmBlue}[1]{{\color{sthlmBlue}{#1}}}
	%\definecolor{sthlmDarkBlue}{RGB}{} %HEX #
		%\newcommand{\csthlmDarkBlue}[1]{{\color{sthlmDarkBlue}{#1}}}

%== GREEN

	\definecolor{sthlmLightGreen}{RGB}{213,247,244} % HEX #0d5f7f4
		\newcommand{\csthlmLightGreen}[1]{{\color{sthlmLightGreen}{#1}}}
	\definecolor{sthlmGreen}{RGB}{0,134,127} % #00867f
		\newcommand{\csthlmGreen}[1]{{\color{sthlmGreen}{#1}}}

%== GREY
	%\definecolor{sthlmLightGrey}{RGB}{}
		%\newcommand{\csthlmLightGrey}[1]{{\color{sthlmLightGrey}{#1}}}
	\definecolor{sthlmGrey}{RGB}{245,243,238} % HEX #f5f3ee
		\newcommand{\csthlmGrey}[1]{{\color{sthlmGrey}{#1}}}
	\definecolor{sthlmDarkGrey}{RGB}{51,51,51} % HEX #333333
		\newcommand{\csthlmDarkGrey}[1]{{\color{sthlmDarkGrey}{#1}}}

%== ORANGE
	\definecolor{sthlmLightOrange}{RGB}{255,215,210} % HEX #ffd7d2
		\newcommand{\csthlmLightOrange}[1]{{\color{sthlmLightOrange}{#1}}}
	\definecolor{sthlmOrange}{RGB}{221,74,44} % HEX #dd4a2c
		\newcommand{\csthlmOrange}[1]{{\color{sthlmOrange}{#1}}}

%== PURPLE

	\definecolor{sthlmLightPurple}{RGB}{241,230,252} % HEX #f1e6fc
		\newcommand{\csthlmLightPurple}[1]{{\color{sthlmLightPurple}{#1}}}
	\definecolor{sthlmPurple}{RGB}{93,35,125} % HEX #5d237d
		\newcommand{\csthlmPurple}[1]{{\color{sthlmPurple}{#1}}}

%== RED

	\definecolor{sthlmLightRed}{RGB}{254,222,237} % HEX #c40064
		\newcommand{\csthlmLightRed}[1]{{\color{sthlmLightRed}{#1}}}
	\definecolor{sthlmRed}{RGB}{196,0,100} % HEX #fedeed
		\newcommand{\csthlmRed}[1]{{\color{sthlmRed}{#1}}}
	%\definecolor{sthlmDarkRed}{RGB}{} % HEX #
		%\newcommand{\csthlmDarkRed}[1]{{\color{sthlmDarkRed}{#1}}}

%== YELLOW

	%\definecolor{sthlmLightYellow}{RGB}{}
		%\newcommand{\csthlmLightYellow}[1]{{\color{sthlmLightYellow}{#1}}}
	\definecolor{sthlmYellow}{RGB}{252,191,10} % HEX #fcbf0a
		\newcommand{\csthlmYellow}[1]{{\color{sthlmYellow}{#1}}}

%-=-=-=-=-=-=-=-=-=-=-=-=-=-=-=-=-=-=-=-=-=-=-=-=
%	ISSR COLORS
%-=-=-=-=-=-=-=-=-=-=-=-=-=-=-=-=-=-=-=-=-=-=-=-=

%== BLUE
	\definecolor{issrBlue}{RGB}{0,111,174} % HEX #0066cc
		\newcommand{\cissrBlue}[1]{{\color{issrBlue}{#1}}}

%== GREY
	\definecolor{issrGrey}{RGB}{167,169,172} % HEX #999999
		\newcommand{\cissrGrey}[1]{{\color{issrGrey}{#1}}}

%-=-=-=-=-=-=-=-=-=-=-=-=-=-=-=-=-=-=-=-=-=-=-=-=
%	GLOBAL COLOR THEME
%-=-=-=-=-=-=-=-=-=-=-=-=-=-=-=-=-=-=-=-=-=-=-=-=

%== BLUE

	\newcommand{\cLightBlue}[1]{{\color{sthlmLightBlue}{#1}}}
	\newcommand{\crLightBlue}{\color{sthlmLightBlue}}
	\newcommand{\cnLightBlue}{sthlmLightBlue}
	\newcommand{\cBlue}[1]{{\color{sthlmBlue}{#1}}}
	\newcommand{\crBlue}{\color{sthlmBlue}}
	\newcommand{\cnBlue}{sthlmBlue}
	%\newcommand{\cDarkBlue}[1]{{\color{}{#1}}}
	%\newcommand{\crDarkBlue}{\color{}}
	%\newcommand{\cnDarkBlue}{}

%== GREEN

	%\newcommand{\cLightGreen}[1]{{\color{}{#1}}}
	%\newcommand{\crLightGreen}{\color{}}
	\newcommand{\cnLightGreen}{sthlmLightGreen}
	\newcommand{\cGreen}[1]{{\color{sthlmGreen}{#1}}}
	\newcommand{\crGreen}{\color{sthlmGreen}}
	\newcommand{\cnGreen}{sthlmGreen}
	%\newcommand{\cDarkGreen}[1]{{\color{}{#1}}}
	%\newcommand{\crDarkGreen}{\color{}}
	%\newcommand{\cnDarkGreen}{sthlmDarkGreen}

%== GREY

	\newcommand{\cLightGrey}[1]{{\color{sthlmLightGrey}{#1}}}
	\newcommand{\crLightGrey}{\color{sthlmLightGrey}}
	\newcommand{\cnLightGrey}{sthlmLightGrey}
	\newcommand{\cGrey}[1]{{\color{sthlmGrey}{#1}}}
	\newcommand{\crGrey}{\color{sthlmGrey}}
	\newcommand{\cnGrey}{sthlmGrey}
	\newcommand{\cDarkGrey}[1]{{\color{sthlmDarkGrey}{#1}}}
	\newcommand{\crDarkGrey}{\color{sthlmDarkGrey}}
	\newcommand{\cnDarkGrey}{sthlmDarkGrey}

%== ORANGE

	\newcommand{\cLightOrange}[1]{{\color{sthlmLightOrange}{#1}}}
	\newcommand{\crLightOrange}{\color{sthlmLightOrange}}
	\newcommand{\cnLightOrange}{sthlmLightOrange}
	\newcommand{\cOrange}[1]{{\color{sthlmOrange}{#1}}}
	\newcommand{\crOrange}{\color{sthlmOrange}}
	\newcommand{\cnOrange}{sthlmOrange}
	%\newcommand{\cDarkOrange}[1]{{\color{}{#1}}}
	%\newcommand{\crDarkOrange}{\color{}}
	%\newcommand{\cnDarkOrange}{}

%== PURPLE

	\newcommand{\cLightPurple}[1]{{\color{sthlmLightPurple}{#1}}}
	\newcommand{\crLightPurple}{\color{sthlmLightPurple}}
	\newcommand{\cnLightPurple}{sthlmLightPurple}
	\newcommand{\cPurple}[1]{{\color{sthlmPurple}{#1}}}
	\newcommand{\crPurple}{\color{sthlmPurple}}
	\newcommand{\cnPurple}{sthlmPurple}
	%\newcommand{\cDarkPurple}[1]{{\color{}{#1}}}
	%\newcommand{\crDarkPurple}{\color{}}
	%\newcommand{\cnDarkPurple}{}

%== RED

	\newcommand{\cLightRed}[1]{{\color{sthlmLightRed}{#1}}}
	\newcommand{\crLightRed}{\color{sthlmLightRed}}
	\newcommand{\cnLightRed}{sthlmLightRed}
	\newcommand{\cRed}[1]{{\color{sthlmRed}{#1}}}
	\newcommand{\crRed}{\color{sthlmRed}}
	\newcommand{\cnRed}{sthlmRed}
	\newcommand{\cDarkRed}[1]{{\color{sthlmDarkRed}{#1}}}
	\newcommand{\crDarkRed}{\color{sthlmDarkRed}}
	\newcommand{\cnDarkRed}{sthlmDarkRed}

%== YELLOW

	%\newcommand{\cLightYellow}[1]{{\color{}{#1}}}
	%\newcommand{\crLightYellow}{\color{}}
	%\newcommand{\cnLightYellow}{sthlmLightYellow}
	\newcommand{\cYellow}[1]{{\color{sthlmYellow}{#1}}}
	\newcommand{\crYellow}{\color{sthlmYellow}}
	\newcommand{\cnYellow}{sthlmYellow}
	%\newcommand{\cDarkYellow}[1]{{\color{}{#1}}}
	%\newcommand{\crDarkYellow}{\color{}}
	%\newcommand{\cnDarkYellow}{}

%-=-=-=-=-=-=-=-=-=-=-=-=-=-=-=-=-=-=-=-=-=-=-=-=
%	GLOBAL UNITS
%-=-=-=-=-=-=-=-=-=-=-=-=-=-=-=-=-=-=-=-=-=-=-=-=

\newcommand{\unit}[1]{\ensuremath{\, \mathrm{#1}}}
\newcommand{\doneit}{\textcolor{sthlmRed}{$\odot$}}

%-=-=-=-=-=-=-=-=-=-=-=-=-=-=-=-=-=-=-=-=-=-=-=-=
%	GLOBAL MATH 20151012-065906
%-=-=-=-=-=-=-=-=-=-=-=-=-=-=-=-=-=-=-=-=-=-=-=-=

\newcommand{\limit}{\displaystyle\lim}
\newcommand{\impart}{\textrm{Im}}
\newcommand{\repart}{\textrm{Re}}
\newcommand{\naturalnum}{\mathbb{N}}
\newcommand{\integernum}{\mathbb{Z}}
\newcommand{\rationalnum}{\mathbb{Q}}
\newcommand{\realnum}{\mathbb{R}}
\newcommand{\complexnum}{\mathbb{C}}
\newcommand{\degree}{\ensuremath{{}^{\circ}}\xspace}
%\newcommand{\degree}{^{\circ}} Depricated by above
\newcommand{\rad}{\unit{rad}}
\newcommand{\dd}{\mathrm{d}}
\newcommand{\dx}{\mathrm{d}x}
\newcommand{\dy}{\mathrm{d}y}
\newcommand{\dr}{\mathrm{d}r}
\newcommand{\dtheta}{\mathrm{d}\theta}
\newcommand{\ddx}{\dfrac{\mathrm{d}}{\mathrm{d}x}}
\newcommand{\dydx}{\dfrac{\mathrm{d}y}{\mathrm{d}x}}
\newcommand{\dydt}{\dfrac{\mathrm{d}y}{\mathrm{d}t}}
\newcommand{\dydu}{\dfrac{\mathrm{d}y}{\mathrm{d}u}}
\newcommand{\dudx}{\dfrac{\mathrm{d}u}{\mathrm{d}x}}
\newcommand{\dxdt}{\dfrac{\mathrm{d}x}{\mathrm{d}t}}
\newcommand{\decf}{\cRed{$\searrow$}}
\newcommand{\incf}{\cGreen{$\nearrow$}}
\newcommand{\stationary}{$\longrightarrow$}
\newcommand{\vect}[1]{\boldsymbol{#1}}
\newcommand{\farg}[1]{\left[\cBlue{\boldsymbol{#1}}\right]}
\newcommand{\set}[1]{\left\{#1\right\}}
\newcommand{\sname}[1]{\texttt{#1}}
\newcommand{\inv}{^{-1}}
\newcommand{\cb}{$\Box$}
\newcommand{\suchthat}{\;\ifnum\currentgrouptype=16 \middle\fi|\;}
\newcommand{\chide}[1]{{\color{sthlmDarkGrey}{#1}}}
\newcommand{\lesssteps}{\cGreen{\textbf{Less Steps Solution:}}}
\newcommand{\solnsteps}{\cGreen{\textbf{Solution:}}}
\newcommand{\qdependlist}{\cDarkGrey{\textbf{Dependencies:}}}
\newcommand{\ihat}{\hat{\imath}}
\newcommand{\jhat}{\hat{\jmath}}
\newcommand{\khat}{\hat{k}}
\DeclareMathOperator{\lcm}{lcm}

\newcommand{\alert}[1]{{\color{sthlmRed}{#1}}}
\newcommand{\mathref}[1]{{\color{sthlmPurple}{#1}}}
\newcommand{\cellyes}{\cellcolor{sthlmGreen!30}Yes}
\newcommand{\cellno}{\cellcolor{sthlmRed!30}No}

\usepackage{multicol}

%-=-=-=-=-=-=-=-=-=-=-=-=-=-=-=-=-=-=-=-=-=-=-=-=
%	FONTS UNITS
%-=-=-=-=-=-=-=-=-=-=-=-=-=-=-=-=-=-=-=-=-=-=-=-=

%\usepackage{avant} % Use the Avantgarde font for headings
%\usepackage{times} % Use the Times font for headings

\usepackage[full]{textcomp}
\usepackage{newpxtext} % osf for text, not math
\usepackage{cabin} % sans serif
\usepackage[varqu,varl]{inconsolata} % sans serif typewriter
\usepackage[bigdelims,vvarbb]{newpxmath} % bb from STIX
\usepackage[cal=boondoxo]{mathalfa} % mathcal

\usepackage{microtype} % Slightly tweak font spacing for aesthetics
\usepackage[utf8]{inputenc} % Required for including letters with accents
\usepackage[T1]{fontenc} % Use 8-bit encoding that has 256 glyphs

%-=-=-=-=-=-=-=-=-=-=-=-=-=-=-=-=-=-=-=-=-=-=-=-=
%	BIBLIOGRAPHY AND INDEX
%-=-=-=-=-=-=-=-=-=-=-=-=-=-=-=-=-=-=-=-=-=-=-=-=
\usepackage[autostyle]{csquotes}
\usepackage[backend=biber,natbib=true,style=alphabetic,citestyle=numeric,sorting=nyt,sortcites=true,autopunct=true,hyperref=true,abbreviate=false,backref=true]{biblatex}
%\addbibresource{bibliography.bib} % BibTeX bibliography file
\defbibheading{bibempty}{}

\usepackage{calc} % For simpler calculation - used for spacing the index letter headings correctly
\usepackage{makeidx} % Required to make an index
\makeindex % Tells LaTeX to create the files required for indexing

%-=-=-=-=-=-=-=-=-=-=-=-=-=-=-=-=-=-=-=-=-=-=-=-=
%	MAIN TABLE OF CONTENTS
%-=-=-=-=-=-=-=-=-=-=-=-=-=-=-=-=-=-=-=-=-=-=-=-=

\usepackage{titletoc} % Required for manipulating the table of contents

\contentsmargin{0cm} % Removes the default margin

% Part text styling
\titlecontents{part}[0cm]
{\addvspace{20pt}\centering\large\bfseries}
{}
{}
{}

% Chapter text styling
\titlecontents{chapter}[1.25cm] % Indentation
{\addvspace{12pt}\large\sffamily\bfseries} % Spacing and font options for chapters
{\color{ocre!60}\contentslabel[\Large\thecontentslabel]{1.25cm}\color{ocre}} % Chapter number
{\color{ocre}}
{\color{ocre!60}\normalsize\;\titlerule*[.5pc]{.}\;\thecontentspage} % Page number

% Section text styling
\titlecontents{section}[1.25cm] % Indentation
{\addvspace{3pt}\sffamily\bfseries} % Spacing and font options for sections
{\contentslabel[\thecontentslabel]{1.25cm}} % Section number
{}
{\hfill\color{black}\thecontentspage} % Page number
[]

% Subsection text styling
\titlecontents{subsection}[1.25cm] % Indentation
{\addvspace{1pt}\sffamily\small} % Spacing and font options for subsections
{\contentslabel[\thecontentslabel]{1.25cm}} % Subsection number
{}
{\ \titlerule*[.5pc]{.}\;\thecontentspage} % Page number
[]

% List of figures
\titlecontents{figure}[0em]
{\addvspace{-5pt}\sffamily}
{\thecontentslabel\hspace*{1em}}
{}
{\ \titlerule*[.5pc]{.}\;\thecontentspage}
[]

% List of tables
\titlecontents{table}[0em]
{\addvspace{-5pt}\sffamily}
{\thecontentslabel\hspace*{1em}}
{}
{\ \titlerule*[.5pc]{.}\;\thecontentspage}
[]

%----------------------------------------------------------------------------------------
%	MINI TABLE OF CONTENTS IN PART HEADS
%----------------------------------------------------------------------------------------

% Chapter text styling
\titlecontents{lchapter}[0em] % Indenting
{\addvspace{15pt}\large\sffamily\bfseries} % Spacing and font options for chapters
{\color{ocre}\contentslabel[\Large\thecontentslabel]{1.25cm}\color{ocre}} % Chapter number
{}
{\color{ocre}\normalsize\sffamily\bfseries\;\titlerule*[.5pc]{.}\;\thecontentspage} % Page number

%% Section text styling
%\titlecontents{lsection}[0em] % Indenting
%{\sffamily\small} % Spacing and font options for sections
%{\contentslabel[\thecontentslabel]{1.25cm}} % Section number
%{}
%{}
%
%% Subsection text styling
%\titlecontents{lsubsection}[.5em] % Indentation
%{\normalfont\footnotesize\sffamily} % Font settings
%{}
%{}
%{}

%-=-=-=-=-=-=-=-=-=-=-=-=-=-=-=-=-=-=-=-=-=-=-=-=
%	PAGE HEADINGS
%-=-=-=-=-=-=-=-=-=-=-=-=-=-=-=-=-=-=-=-=-=-=-=-=

\usepackage{fancyhdr} % Required for header and footer configuration

\pagestyle{fancy}
%\renewcommand{\chaptermark}[1]{\markboth{\sffamily\normalsize\bfseries\chaptername\ \thechapter.\ #1}{}}
\renewcommand{\chaptermark}[1]{\markboth{\sffamily\normalsize\bfseries  \thechapter.\ #1}{}} % Chapter text font settings
\renewcommand{\sectionmark}[1]{\markright{\sffamily\normalsize\thesection\hspace{5pt}#1}{}} % Section text font settings
\fancyhf{} \fancyhead[LE,RO]{\sffamily\normalsize\thepage} % Font setting for the page number in the header
\fancyhead[LO]{\rightmark} % Print the nearest section name on the left side of odd pages
\fancyhead[RE]{\leftmark} % Print the current chapter name on the right side of even pages
\renewcommand{\headrulewidth}{0.5pt} % Width of the rule under the header
\addtolength{\headheight}{2.5pt} % Increase the spacing around the header slightly
\renewcommand{\footrulewidth}{0pt} % Removes the rule in the footer
\fancypagestyle{plain}{\fancyhead{}\renewcommand{\headrulewidth}{0pt}} % Style for when a plain pagestyle is specified

% Removes the header from odd empty pages at the end of chapters
\makeatletter
\renewcommand{\cleardoublepage}{
\clearpage\ifodd\c@page\else
\hbox{}
\vspace*{\fill}
\thispagestyle{empty}
\newpage
\fi}

%-=-=-=-=-=-=-=-=-=-=-=-=-=-=-=-=-=-=-=-=-=-=-=-=
%	THEOREM STYLES
%-=-=-=-=-=-=-=-=-=-=-=-=-=-=-=-=-=-=-=-=-=-=-=-=

\newcommand{\intoo}[2]{\mathopen{]}#1\,;#2\mathclose{[}}
\newcommand{\ud}{\mathop{\mathrm{{}d}}\mathopen{}}
\newcommand{\intff}[2]{\mathopen{[}#1\,;#2\mathclose{]}}
\newtheorem{notation}{Notation}[chapter]

\newtheoremstyle{ocrenumbox}% % Theorem style name
{0pt}% Space above
{0pt}% Space below
{\normalfont}% % Body font
{}% Indent amount
{\small\bf\sffamily\color{ocre}}% % Theorem head font
{\;}% Punctuation after theorem head
{0.25em}% Space after theorem head
{\small\sffamily\color{ocre}\thmname{#1}\nobreakspace\thmnumber{\@ifnotempty{#1}{}\@upn{#2}}% Theorem text (e.g. Theorem 2.1)
\thmnote{\nobreakspace\the\thm@notefont\sffamily\bfseries\color{black}---\nobreakspace#3.}} % Optional theorem note
\renewcommand{\qedsymbol}{$\blacksquare$}% Optional qed square

\newtheoremstyle{blacknumexa}% Theorem style Example
{5pt}% Space above
{5pt}% Space below
{\normalfont}% Body font
{} % Indent amount
{\small\bf\sffamily\color{sthlmGreen}}% Theorem head font
{\;}% Punctuation after theorem head
{0.25em}% Space after theorem head
{\nobreakspace\thmname{#1}\nobreakspace\thmnumber{\@ifnotempty{#1}{}\@upn{#2}}% Theorem text (e.g. Theorem 2.1)
\thmnote{\nobreakspace\the\thm@notefont\sffamily\bfseries---\nobreakspace#3.}}% Optional theorem note

\newtheoremstyle{blacknumexe}% Theorem style Exercise
{5pt}% Space above
{5pt}% Space below
{\normalfont}% Body font
{} % Indent amount
{\small\bf\sffamily\color{ocre}}% Theorem head font
{\;}% Punctuation after theorem head
{0.25em}% Space after theorem head
{\nobreakspace\thmname{#1}\nobreakspace\thmnumber{\@ifnotempty{#1}{}\@upn{#2}}% Theorem text (e.g. Theorem 2.1)
\thmnote{\nobreakspace\the\thm@notefont\sffamily\bfseries---\nobreakspace#3.}}% Optional theorem note

\newtheoremstyle{blacknumessential}% Theorem style name
{5pt}% Space above
{5pt}% Space below
{\normalfont}% Body font
{} % Indent amount
{\small\bf\sffamily\color{sthlmRed}}% Theorem head font
{\;}% Punctuation after theorem head
{0.25em}% Space after theorem head
{\nobreakspace\thmname{#1}\nobreakspace\thmnumber{\@ifnotempty{#1}{}\@upn{#2}}% Theorem text (e.g. Theorem 2.1)
\thmnote{\nobreakspace\the\thm@notefont\sffamily\bfseries---\nobreakspace#3.}}% Optional theorem note

\newtheoremstyle{blacknumpro}% Theorem style name properties
{5pt}% Space above
{5pt}% Space below
{\normalfont}% Body font
{} % Indent amount
{\small\bf\sffamily\color{sthlmPurple}}% Theorem head font
{\;}% Punctuation after theorem head
{0.25em}% Space after theorem head
{\nobreakspace\thmname{#1}\nobreakspace\thmnumber{\@ifnotempty{#1}{}\@upn{#2}}% Theorem text (e.g. Theorem 2.1)
\thmnote{\nobreakspace\the\thm@notefont\sffamily\bfseries---\nobreakspace#3.}}% Optional theorem note

\newtheoremstyle{blacknumbox} % Theorem style name
{0pt}% Space above
{0pt}% Space below
{\normalfont}% Body font
{}% Indent amount
{\small\bf\sffamily}% Theorem head font
{\;}% Punctuation after theorem head
{0.25em}% Space after theorem head
{\small\sffamily\thmname{#1}\nobreakspace\thmnumber{\@ifnotempty{#1}{}\@upn{#2}}% Theorem text (e.g. Theorem 2.1)
\thmnote{\nobreakspace\the\thm@notefont\sffamily\bfseries---\nobreakspace#3.}}% Optional theorem note

% Non-boxed/non-framed environments
\newtheoremstyle{ocrenum}% % Theorem style name
{5pt}% Space above
{5pt}% Space below
{\normalfont}% % Body font
{}% Indent amount
{\small\bf\sffamily\color{ocre}}% % Theorem head font
{\;}% Punctuation after theorem head
{0.25em}% Space after theorem head
{\small\sffamily\color{ocre}\thmname{#1}\nobreakspace\thmnumber{\@ifnotempty{#1}{}\@upn{#2}}% Theorem text (e.g. Theorem 2.1)
\thmnote{\nobreakspace\the\thm@notefont\sffamily\bfseries\color{black}---\nobreakspace#3.}} % Optional theorem note
\renewcommand{\qedsymbol}{$\blacksquare$}% Optional qed square
\makeatother

% Defines the theorem text style for each type of theorem to one of the three styles above
\newcounter{dummy}
\numberwithin{dummy}{section}
\theoremstyle{ocrenumbox}
\newtheorem{theoremeT}[dummy]{Theorem}
\newtheorem{aruleT}[dummy]{Rule}
\newtheorem{aidentityT}[dummy]{Identity}
\newtheorem{problem}{Problem}[chapter]
\theoremstyle{blacknumexe}
\newtheorem{exerciseT}{Exercise}[chapter]
\theoremstyle{blacknumexa}
\newtheorem{exampleT}{Example}[chapter]
\theoremstyle{blacknumessential}
\newtheorem{essentialqT}{Essential Questions}[chapter]
\theoremstyle{blacknumpro}
\newtheorem{propertyT}[dummy]{Property}
\theoremstyle{blacknumbox}
\newtheorem{notationT}[dummy]{Notation}
\theoremstyle{blacknumbox}
\newtheorem{vocabulary}{Vocabulary}[chapter]
\newtheorem{definitionT}{Definition}[section]
\newtheorem{corollaryT}[dummy]{Corollary}
\theoremstyle{ocrenum}
\newtheorem{proposition}[dummy]{Proposition}
\newtheorem{qsT}{Q}[chapter]
%----------------------------------------------------------------------------------------
%	DEFINITION OF COLORED BOXES
%----------------------------------------------------------------------------------------

\RequirePackage[framemethod=default]{mdframed} % Required for creating the theorem, definition, exercise and corollary boxes

% Theorem box
\newmdenv[skipabove=7pt,
skipbelow=7pt,
backgroundcolor=black!5,
linecolor=ocre,
innerleftmargin=5pt,
innerrightmargin=5pt,
innertopmargin=5pt,
leftmargin=0cm,
rightmargin=0cm,
innerbottommargin=5pt,
linewidth=2pt]{tBox}

% Rule box
\newmdenv[skipabove=7pt,
skipbelow=7pt,
backgroundcolor=black!5,
linecolor=ocre,
innerleftmargin=5pt,
innerrightmargin=5pt,
innertopmargin=5pt,
leftmargin=0cm,
rightmargin=0cm,
innerbottommargin=5pt,
linewidth=2pt]{arBox}

% Identity box
\newmdenv[skipabove=7pt,
skipbelow=7pt,
linecolor=gray,
backgroundcolor=black!5,
innerleftmargin=5pt,
innerrightmargin=5pt,
innertopmargin=5pt,
leftmargin=0cm,
rightmargin=0cm,
innerbottommargin=5pt,
linewidth=2pt]{idenBox}

% Notation box
\newmdenv[skipabove=7pt,
skipbelow=7pt,
backgroundcolor=sthlmOrange!5,
linecolor=sthlmOrange,
innerleftmargin=5pt,
innerrightmargin=5pt,
innertopmargin=5pt,
leftmargin=0cm,
rightmargin=0cm,
innerbottommargin=5pt,
linewidth=2pt]{notBox}

% Definition box
\newmdenv[skipabove=7pt,
skipbelow=7pt,
backgroundcolor=sthlmDarkGrey!5,
linecolor=ocre,
innerleftmargin=5pt,
innerrightmargin=5pt,
innertopmargin=5pt,
innerbottommargin=5pt,
leftmargin=0cm,
rightmargin=0cm,
innerbottommargin=5pt,
linewidth=2pt]{dBox}

% Property box
\newmdenv[skipabove=7pt,
skipbelow=7pt,
backgroundcolor=sthlmDarkGrey!5,
linecolor=sthlmPurple,
innerleftmargin=5pt,
innerrightmargin=5pt,
innertopmargin=5pt,
leftmargin=0cm,
rightmargin=0cm,
innerbottommargin=5pt,
linewidth=2pt]{prBox}

% Exercise box
\newmdenv[skipabove=7pt,
skipbelow=7pt,
rightline=false,
leftline=true,
topline=false,
bottomline=false,
backgroundcolor=ocre!5,
linecolor=ocre,
innerleftmargin=5pt,
innerrightmargin=5pt,
innertopmargin=5pt,
innerbottommargin=5pt,
leftmargin=0cm,
rightmargin=0cm,
linewidth=4pt]{eBox}

% Example box
\newmdenv[skipabove=7pt,
skipbelow=7pt,
rightline=false,
leftline=true,
topline=false,
bottomline=false,
backgroundcolor=sthlmGreen!5,
linecolor=sthlmGreen,
innerleftmargin=5pt,
innerrightmargin=5pt,
innertopmargin=5pt,
innerbottommargin=5pt,
leftmargin=0cm,
rightmargin=0cm,
linewidth=4pt]{exaBox}

% Essential Question box
\newmdenv[skipabove=7pt,
skipbelow=7pt,
rightline=false,
leftline=false,
topline=true,
bottomline=true,
backgroundcolor=sthlmRed!5,
linecolor=sthlmRed,
innerleftmargin=5pt,
innerrightmargin=5pt,
innertopmargin=5pt,
innerbottommargin=5pt,
leftmargin=0cm,
rightmargin=0cm,
linewidth=4pt]{essentialqBox}

% Corollary box
\newmdenv[skipabove=7pt,
skipbelow=7pt,
rightline=false,
leftline=true,
topline=false,
bottomline=false,
linecolor=gray,
backgroundcolor=black!5,
innerleftmargin=5pt,
innerrightmargin=5pt,
innertopmargin=5pt,
leftmargin=0cm,
rightmargin=0cm,
linewidth=4pt,
innerbottommargin=5pt]{cBox}

% Creates an environment for each type of theorem and assigns it a theorem text style from the "Theorem Styles" section above and a colored box from above
\newenvironment{theorem}{\begin{tBox}\begin{theoremeT}}{\end{theoremeT}\end{tBox}}
\newenvironment{arule}{\begin{arBox}\begin{aruleT}}{\end{aruleT}\end{arBox}}
\newenvironment{aidentity}{\begin{idenBox}\begin{aidentityT}}{\end{aidentityT}\end{idenBox}}
\newenvironment{exercise}{\begin{eBox}\begin{exerciseT}}{\hfill{\color{ocre}\tiny\ensuremath{\blacksquare}}\end{exerciseT}\end{eBox}}
\newenvironment{definition}{\begin{dBox}\begin{definitionT}}{\end{definitionT}\end{dBox}}
\newenvironment{property}{\begin{prBox}\begin{propertyT}}{\end{propertyT}\end{prBox}}
\newenvironment{notations}{\begin{notBox}\begin{notationT}}{\end{notationT}\end{notBox}}
\newenvironment{example}{\begin{exaBox}\begin{exampleT}}{\hfill{\color{sthlmGreen}\tiny\ensuremath{\blacksquare}}\end{exampleT}\end{exaBox}}
\newenvironment{essentialq}{\begin{essentialqBox}\begin{essentialqT}}{\end{essentialqT}\end{essentialqBox}}
\newenvironment{corollary}{\begin{cBox}\begin{corollaryT}}{\end{corollaryT}\end{cBox}}
\newenvironment{qst}{\begin{essentialqBox}\begin{qsT}}{\end{qsT}\end{essentialqBox}}
\newtheorem{jd}{A:}
\newcommand{\ans}{{\small\bf\sffamily\color{ocre}A:}\; }
%----------------------------------------------------------------------------------------
%	REMARK ENVIRONMENT
%----------------------------------------------------------------------------------------

\newenvironment{remark}{\par\vspace{10pt}\small % Vertical white space above the remark and smaller font size
\begin{list}{}{
\leftmargin=35pt % Indentation on the left
\rightmargin=25pt}\item\ignorespaces % Indentation on the right
\makebox[-2.5pt]{\begin{tikzpicture}[overlay]
\node[draw=ocre!60,line width=1pt,circle,fill=ocre!25,font=\sffamily\bfseries,inner sep=2pt,outer sep=0pt] at (-15pt,0pt){\textcolor{ocre}{R}};\end{tikzpicture}} % Orange R in a circle
\advance\baselineskip -1pt}{\end{list}\vskip5pt} % Tighter line spacing and white space after remark

\newcommand{\soln}[1]{\par\vspace{10pt}\makebox[15pt]{\begin{tikzpicture}[overlay]
\node[draw=sthlmGreen!70,line width=1pt,circle,fill=sthlmGreen!25,font=\sffamily\bfseries,inner sep=2pt,outer sep=0pt] at (-15pt,0pt){\textcolor{sthlmGreen}{S}};\end{tikzpicture}}\cGreen{\noindent\hrulefill\vskip 15pt}}

\newcommand{\qdepend}[1]{\par\vspace{10pt}\makebox[15pt]{\begin{tikzpicture}[overlay]
\node[draw=sthlmDarkGrey!70,line width=1pt,circle,fill=sthlmDarkGrey!25,font=\sffamily\bfseries,inner sep=2pt,outer sep=0pt] at (-15pt,0pt){\textcolor{sthlmDarkGrey}{D}};\end{tikzpicture}}\cDarkGrey{\noindent\hrulefill\vskip 15pt}}

%-=-=-=-=-=-=-=-=-=-=-=-=-=-=-=-=-=-=-=-=-=-=-=-=
%	SECTION NUMBERING IN THE MARGIN
%-=-=-=-=-=-=-=-=-=-=-=-=-=-=-=-=-=-=-=-=-=-=-=-=

\makeatletter
\renewcommand{\@seccntformat}[1]{\llap{\textcolor{ocre}{\csname the#1\endcsname}\hspace{1em}}}
\renewcommand{\section}{\@startsection{section}{1}{\z@}
{-4ex \@plus -1ex \@minus -.4ex}
{1ex \@plus.2ex }
{\normalfont\large\sffamily\bfseries}}
\renewcommand{\subsection}{\@startsection {subsection}{2}{\z@}
{-3ex \@plus -0.1ex \@minus -.4ex}
{0.5ex \@plus.2ex }
{\normalfont\sffamily\bfseries}}
\renewcommand{\subsubsection}{\@startsection {subsubsection}{3}{\z@}
{-2ex \@plus -0.1ex \@minus -.2ex}
{.2ex \@plus.2ex }
{\normalfont\small\sffamily\bfseries}}
\renewcommand\paragraph{\@startsection{paragraph}{4}{\z@}
{-2ex \@plus-.2ex \@minus .2ex}
{.1ex}
{\normalfont\small\sffamily\bfseries}}

%-=-=-=-=-=-=-=-=-=-=-=-=-=-=-=-=-=-=-=-=-=-=-=-=
%	PART HEADINGS
%-=-=-=-=-=-=-=-=-=-=-=-=-=-=-=-=-=-=-=-=-=-=-=-=

% numbered part in the table of contents
\newcommand{\@mypartnumtocformat}[2]{%
\setlength\fboxsep{0pt}%
\noindent\colorbox{ocre!20}{\strut\parbox[c][.7cm]{\ecart}{\color{ocre!70}\Large\sffamily\bfseries\centering#1}}\hskip\esp\colorbox{ocre!40}{\strut\parbox[c][.7cm]{\linewidth-\ecart-\esp}{\Large\sffamily\centering#2}}}%
%%%%%%%%%%%%%%%%%%%%%%%%%%%%%%%%%%
% unnumbered part in the table of contents
\newcommand{\@myparttocformat}[1]{%
\setlength\fboxsep{0pt}%
\noindent\colorbox{ocre!40}{\strut\parbox[c][.7cm]{\linewidth}{\Large\sffamily\centering#1}}}%
%%%%%%%%%%%%%%%%%%%%%%%%%%%%%%%%%%
\newlength\esp
\setlength\esp{4pt}
\newlength\ecart
\setlength\ecart{1.2cm-\esp}
\newcommand{\thepartimage}{}%
\newcommand{\partimage}[1]{\renewcommand{\thepartimage}{#1}}%
\def\@part[#1]#2{%
\ifnum \c@secnumdepth >-2\relax%
\refstepcounter{part}%
\addcontentsline{toc}{part}{\texorpdfstring{\protect\@mypartnumtocformat{\thepart}{#1}}{\partname~\thepart\ ---\ #1}}
\else%
\addcontentsline{toc}{part}{\texorpdfstring{\protect\@myparttocformat{#1}}{#1}}%
\fi%
\startcontents%
\markboth{}{}%
{\thispagestyle{empty}%
\begin{tikzpicture}[remember picture,overlay]%
\node at (current page.north west){\begin{tikzpicture}[remember picture,overlay]%
\fill[ocre!20](0cm,0cm) rectangle (\paperwidth,-\paperheight);
\node[anchor=north] at (4cm,-3.25cm){\color{ocre!40}\fontsize{220}{100}\sffamily\bfseries\@Roman\c@part};
\node[anchor=south east] at (\paperwidth-1cm,-\paperheight+1cm){\parbox[t][][t]{8.5cm}{
\printcontents{l}{0}{\setcounter{tocdepth}{1}}%
}};
\node[anchor=north east] at (\paperwidth-1.5cm,-3.25cm){\parbox[t][][t]{15cm}{\strut\raggedleft\color{white}\fontsize{30}{30}\sffamily\bfseries#2}};
\end{tikzpicture}};
\end{tikzpicture}}%
\@endpart}
\def\@spart#1{%
\startcontents%
\phantomsection
{\thispagestyle{empty}%
\begin{tikzpicture}[remember picture,overlay]%
\node at (current page.north west){\begin{tikzpicture}[remember picture,overlay]%
\fill[ocre!20](0cm,0cm) rectangle (\paperwidth,-\paperheight);
\node[anchor=north east] at (\paperwidth-1.5cm,-3.25cm){\parbox[t][][t]{15cm}{\strut\raggedleft\color{white}\fontsize{30}{30}\sffamily\bfseries#1}};
\end{tikzpicture}};
\end{tikzpicture}}
\addcontentsline{toc}{part}{\texorpdfstring{%
\setlength\fboxsep{0pt}%
\noindent\protect\colorbox{ocre!40}{\strut\protect\parbox[c][.7cm]{\linewidth}{\Large\sffamily\protect\centering #1\quad\mbox{}}}}{#1}}%
\@endpart}
\def\@endpart{\vfil\newpage
\if@twoside
\if@openright
\null
\thispagestyle{empty}%
\newpage
\fi
\fi
\if@tempswa
\twocolumn
\fi}

%-=-=-=-=-=-=-=-=-=-=-=-=-=-=-=-=-=-=-=-=-=-=-=-=
%	CHAPTER HEADINGS
%-=-=-=-=-=-=-=-=-=-=-=-=-=-=-=-=-=-=-=-=-=-=-=-=

%\newcommand{\thechapterimage}{}%
%\newcommand{\chapterimage}[1]{\renewcommand{\thechapterimage}{#1}}%
\def\@makechapterhead#1{%
{\parindent \z@ \raggedright \normalfont
\ifnum \c@secnumdepth >\m@ne
\if@mainmatter
\begin{tikzpicture}[remember picture,overlay]
\node at (current page.north west)
{\begin{tikzpicture}[remember picture,overlay]
%\node[anchor=north west,inner sep=0pt] at (0,0) {\includegraphics[width=\paperwidth]{\thechapterimage}};
\draw[anchor=west] (\Gm@lmargin,-4cm) node [line width=2pt,rounded corners=15pt,draw=ocre,fill=white,fill opacity=0.5,inner sep=15pt]{\strut\makebox[22cm]{}};
\draw[anchor=west] (\Gm@lmargin+.3cm,-4cm) node {\huge\sffamily\bfseries\color{black}\thechapter. #1\strut};
\end{tikzpicture}};
\end{tikzpicture}
\else
\begin{tikzpicture}[remember picture,overlay]
\node at (current page.north west)
{\begin{tikzpicture}[remember picture,overlay]
%\node[anchor=north west,inner sep=0pt] at (0,0) {\includegraphics[width=\paperwidth]{\thechapterimage}};
\draw[anchor=west] (\Gm@lmargin,-4cm) node [line width=2pt,rounded corners=15pt,draw=ocre,fill=white,fill opacity=0.5,inner sep=15pt]{\strut\makebox[22cm]{}};
\draw[anchor=west] (\Gm@lmargin+.3cm,-4cm) node {\huge\sffamily\bfseries\color{black}#1\strut};
\end{tikzpicture}};
\end{tikzpicture}
\fi\fi\par\vspace*{100\p@}}}

%-------------------------------------------

\def\@makeschapterhead#1{%
\begin{tikzpicture}[remember picture,overlay]
\node at (current page.north west)
{\begin{tikzpicture}[remember picture,overlay]
%\node[anchor=north west,inner sep=0pt] at (0,0) {\includegraphics[width=\paperwidth]{\thechapterimage}};
\draw[anchor=west] (\Gm@lmargin,-4cm) node [line width=2pt,rounded corners=15pt,draw=ocre,fill=white,fill opacity=0.5,inner sep=15pt]{\strut\makebox[22cm]{}};
\draw[anchor=west] (\Gm@lmargin+.3cm,-4cm) node {\huge\sffamily\bfseries\color{black}#1\strut};
\end{tikzpicture}};
\end{tikzpicture}
\par\vspace*{100\p@}}
\makeatother

%-=-=-=-=-=-=-=-=-=-=-=-=-=-=-=-=-=-=-=-=-=-=-=-=
%	HYPERLINKS IN THE DOCUMENTS
%-=-=-=-=-=-=-=-=-=-=-=-=-=-=-=-=-=-=-=-=-=-=-=-=

\usepackage{hyperref}
\hypersetup{hidelinks,backref=true,pagebackref=true,hyperindex=true,colorlinks=false,breaklinks=true,urlcolor= ocre,bookmarks=true,bookmarksopen=false,pdftitle={Title},pdfauthor={Author}}
\usepackage{bookmark}
\bookmarksetup{
open,
numbered,
addtohook={%
\ifnum\bookmarkget{level}=0 % chapter
\bookmarksetup{bold}%
\fi
\ifnum\bookmarkget{level}=-1 % part
\bookmarksetup{color=ocre,bold}%
\fi
}
}

%-=-=-=-=-=-=-=-=-=-=-=-=-=-=-=-=-=-=-=-=-=-=-=-=
%	TIKZ LIBRARIES
%-=-=-=-=-=-=-=-=-=-=-=-=-=-=-=-=-=-=-=-=-=-=-=-=

\usetikzlibrary{
arrows,
backgrounds,
fit,
chains,
calc,
decorations.pathmorphing,
matrix,
mindmap,
positioning,
shapes,
decorations,
shadows,
trees
}

\tikzset{ >=stealth', help lines/.style={dashed, thick}, axis/.style={<->}, important line/.style={thick}, connection/.style={thick, dotted},}

%-=-=-=-=-=-=-=-=-=-=-=-=-=-=-=-=-=-=-=-=-=-=-=-=
%	TIKZ FACTOR
%-=-=-=-=-=-=-=-=-=-=-=-=-=-=-=-=-=-=-=-=-=-=-=-=
\tikzstyle{firstterm} = [circle, draw, fill=sthlmRed!20, text centered,minimum size=1cm]
\tikzstyle{secondterm} = [circle, draw, fill=sthlmBlue!20, text centered, node distance=0.25cm,minimum size=1cm]
\tikzstyle{terms} = [rectangle, draw, fill=sthlmDarkGrey!20, text centered, node distance=0.25cm,minimum size=1cm]
\tikzstyle{factoradd} = [rectangle, draw=none, text centered, node distance=0.25cm,minimum size=1cm]
\tikzstyle{multiply} = [rectangle, draw, fill=sthlmDarkGrey!20, text centered, minimum height=0.75cm,minimum size=1cm]
\tikzstyle{add} = [rectangle, draw, fill=sthlmGreen!20, text centered, minimum height=0.75cm,minimum size=1cm]
\tikzstyle{line} = [draw, color=black!50, -latex']
% Define the style for the red dotted boxes
\tikzset{factordotted/.style={draw=sthlmPurple!50!white, line width=1pt, dash pattern=on 1pt off 4pt on 6pt off 4pt, inner sep=4mm, rectangle, rounded corners}}

%-=-=-=-=-=-=-=-=-=-=-=-=-=-=-=-=-=-=-=-=-=-=-=-=
%	WORKFLOW SETUP: SIMPLIFYING EXPRESSIONS
%-=-=-=-=-=-=-=-=-=-=-=-=-=-=-=-=-=-=-=-=-=-=-=-=

\tikzstyle{operation} = [rectangle, draw, fill=sthlmBlue!40, text centered, rounded corners]
\tikzstyle{function} = [rectangle, draw, fill=sthlmPurple!40, text centered, rounded corners]
\tikzstyle{notation} = [rectangle, draw, fill=sthlmGreen!40, text centered, rounded corners]
\tikzstyle{algorithm} = [rectangle, draw, fill=sthlmOrange!40, text centered, rounded corners]
\tikzstyle{delim} = [rectangle, draw, fill=sthlmYellow!40, text centered, rounded corners]
\tikzstyle{property} = [rectangle, draw, fill=sthlmRed!40, text centered, rounded corners]
\tikzstyle{hide} = [rectangle, draw, fill=sthlmGrey, text centered, rounded corners]
\tikzstyle{active} = [rectangle, draw, fill=sthlmDarkGrey, text=white, text centered, rounded corners]

\tikzset{functiondotted/.style={draw=sthlmPurple!50!white, line width=1pt,
                               dash pattern=on 1pt off 4pt on 6pt off 4pt,
                                inner sep=4mm, rectangle, rounded corners}}

\tikzset{productdotted/.style={draw=sthlmBlue!50!white, line width=1pt,
                               dash pattern=on 1pt off 4pt on 6pt off 4pt,
                                inner sep=4mm, rectangle, rounded corners}}

\tikzset{sumdotted/.style={draw=sthlmRed!50!white, line width=1pt,
                               dash pattern=on 1pt off 4pt on 6pt off 4pt,
                                inner sep=4mm, rectangle, rounded corners}}

%-=-=-=-=-=-=-=-=-=-=-=-=-=-=-=-=-=-=-=-=-=-=-=-=
%	WORKFLOW SETUP TRIGONOMETRIC EQUATIONS
%-=-=-=-=-=-=-=-=-=-=-=-=-=-=-=-=-=-=-=-=-=-=-=-=

\tikzstyle{decision} = [diamond, draw, fill=sthlmGreen!20,
    text width=4.5em, text badly centered, node distance=1.2cm, inner sep=0pt]
\tikzstyle{abscissa} = [rectangle, draw, fill=sthlmBlue!20,
    text width=8em, text centered, rounded corners, node distance=1.2cm, minimum height=2.8em]
\tikzstyle{ordinate} = [rectangle, draw, fill=sthlmRed!20,
    text width=8em, text centered, rounded corners, node distance=1.2cm, minimum height=2.8em]
\tikzstyle{task} = [rectangle, draw, fill=sthlmPurple!20,
   minimum width=8em, text centered, rounded corners, node distance=1.2cm, minimum height=2.8em]
\tikzstyle{pointcircle} = [circle, draw, fill=sthlmDarkGrey!20, text centered, node distance=1cm]
% Define the style for the red dotted boxes
  \tikzset{bluedotted/.style={draw=sthlmBlue!50!white, line width=1pt,
                               dash pattern=on 1pt off 4pt on 6pt off 4pt,
                                inner sep=4mm, rectangle, rounded corners}}
% Define the style for the red dotted boxes
  \tikzset{reddotted/.style={draw=sthlmRed!50!white, line width=1pt,
                               dash pattern=on 1pt off 4pt on 6pt off 4pt,
                                inner sep=4mm, rectangle, rounded corners}}

%-=-=-=-=-=-=-=-=-=-=-=-=-=-=-=-=-=-=-=-=-=-=-=-=
%	Derivative Workflow
%-=-=-=-=-=-=-=-=-=-=-=-=-=-=-=-=-=-=-=-=-=-=-=-=	
\tikzstyle{point} = [rectangle, draw, fill=sthlmBlue!40, text centered,rounded corners]
\tikzstyle{limit} = [rectangle, draw, fill=sthlmPurple!40, text centered,rounded corners]
\tikzstyle{class} = [rectangle, draw, fill=sthlmGreen!40, text centered, rounded corners]
\tikzstyle{intervals} = [rectangle, draw, fill=sthlmOrange!40, text centered, rounded corners]
\tikzstyle{fd} = [rectangle, draw, fill=sthlmRed!40, text centered, rounded corners]
\tikzstyle{secondderivative} = [circle, draw, fill=sthlmGrey,radius=0.5, text width=1em, node distance =2.5cm]

%-=-=-=-=-=-=-=-=-=-=-=-=-=-=-=-=-=-=-=-=-=-=-=-=
%	General Workflow
%-=-=-=-=-=-=-=-=-=-=-=-=-=-=-=-=-=-=-=-=-=-=-=-=	
\tikzstyle{bluerbox} = [rectangle, draw, fill=sthlmBlue!40, text centered,rounded corners]
\tikzstyle{purplerbox} = [rectangle, draw, fill=sthlmPurple!40, text centered,rounded corners]
\tikzstyle{greenrbox} = [rectangle, draw, fill=sthlmGreen!40, text centered, rounded corners]
\tikzstyle{orangerbox} = [rectangle, draw, fill=sthlmOrange!40, text centered, rounded corners]
\tikzstyle{redrbox} = [rectangle, draw, fill=sthlmRed!40, text centered, rounded corners]
\tikzstyle{circlenode} = [circle, draw, fill=sthlmGrey,radius=0.5, text width=1em, node distance =2.5cm]
\tikzstyle{line} = [draw, very thick, color=black!50, -latex']
\tikzstyle{dashline} = [draw, very thick, dash pattern=on 1pt off 4pt on 6pt off 4pt,inner sep=4mm,color=black!50, -latex']

%\allowdisplaybreaks[4]   %%%%%允许多行公式中间换页, [n]有1~4可选,表示换页的允许程度,
%%%%%%4为默认值, \\*表示不允许在此处换页
%                         %%%%%也可以在换行命令\\前临时插入\displaybreak[n]来实现,
%                         %%%%%n有0~4可选,0表示可以换页但尽量避免
%
\newcommand{\R}{\mathbb{R}}
\newcommand{\N}{\mathbb{N}}
\renewcommand{\vec}[1]{\mbox{\boldmath $#1$}}      %定义向量为粗斜体
\newcommand{\me}{\mathrm{e}}                       %定义无理常数e为直立体
\newcommand{\mi}{\mathrm{i}}                       %定义虚数i为直立体
\newcommand{\dif}{\mathrm{d}}                      %定义微分算子d为直立体
%\DeclareSymbolFont{lettersA}{U}{pxmia}{m}{it}      %定义\pi 为直立体
%\DeclareMathSymbol{\piup}{\mathord}{lettersA}{"19} %定义\pi 为直立体
%
\newtheorem{thm}{\hspace*{2.0em}定理\hspace*{0.1em}}%[section]
\newtheorem{cor}{\hspace*{2.0em}推论\hspace*{0.1em}}[thm]

%\theoremstyle{nonumberplain}
%\theorembodyfont{\rmfamily}
%\theoremheaderfont{\sffamily}
%\theoremsymbol{\ensuremath{\blacksquare}}
%\theoremseparator{\,}
%\newtheorem{proof}{\indent 证明}

%\renewenvironment{proof}{begin}{end}
%\renewcommand{\proofname}{证明}

\newcommand\raiseeqn{%
\setlength{\abovedisplayskip}{-\baselineskip}%
\setlength{\abovedisplayshortskip}{-\baselineskip}}
\newenvironment{z}{%
\raiseeqn
\csuse{align*}%
}{%
\csuse{endalign*}%
}

\DeclareMathOperator{\Prob}{Prob}
\DeclareMathOperator{\Arg}{Arg}
\DeclareMathOperator{\sgn}{sgn}
\newcommand*\upcite[1]{\textsuperscript{\cite{#1}}}
\numberwithin{equation}{section}
